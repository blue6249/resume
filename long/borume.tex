\documentclass[10pt]{article}
\usepackage{fontspec} 

\setlength\parindent{0in}

\topmargin 0pt
\headheight 0pt
\oddsidemargin 0pt
\evensidemargin 0pt
\usepackage[cm]{fullpage}
\pagestyle{empty}

\usepackage[usenames,dvipsnames]{color}
\usepackage{xunicode}
\usepackage{xltxtra}
\defaultfontfeatures{Mapping=tex-text}
%\setromanfont [Ligatures={Common}, Numbers={OldStyle}, Variant=01]{Linux Libertine O}

\setromanfont [Ligatures={NoRequired,NoCommon,NoContextual}, Numbers={Lining}, Variant=01]{Linux Libertine O}

% HEADINGS
\usepackage{sectsty} 
\usepackage[normalem]{ulem} 
\sectionfont{\mdseries\upshape\Large}
\subsectionfont{\mdseries\upshape\large} 
\subsubsectionfont{\mdseries\slshape\normalsize} 

\usepackage[dvipdfm, bookmarks, colorlinks, breaklinks, 
	pdftitle={James McGuire},
	pdfauthor={James McGuire}
]{hyperref}  
\hypersetup{linkcolor=blue,citecolor=blue,filecolor=black,urlcolor=MidnightBlue} 

% DOCUMENT
\begin{document}
{\LARGE James McGuire}\\[0.5cm]
12235 Vance Jackson Rd. Apt. 1711\\
San Antonio, TX \texttt{78230}\\[.2cm]
\texttt{479-633-7565}\\
\href{mailto:james@sadbox.org}{james@sadbox.org}\\

\section*{Profile}
I am a Systems Administrator that has experience administering Unix, Linux, Windows, and VMWare servers in an enterprise environment. In each role I always placed a focus on following best practices and providing my team with the documentation and tools necessary to perform their jobs better. I work primarily on Linux based systems and develop tools that run on them.

\section*{Core Technical Skills}
\textbf{Systems:} ESX/VMWare, HP-UX, SuSE Enterprise Linux, Debian and Redhat Based Distributions, Windows Server 2003 and 2008 \\
\textbf{Languages:} Python, Perl, Bash/ksh scripting \\
\textbf{Specialties:} Documentation, Automation, New-hire training, Enterprise hardware/software installs, Desktop support \\

\section*{Experience}
\subsection*{NOC Engineer - Consert \hfill September 2012-Present}
\subsubsection*{Development}
\begin{itemize}
    \item Wrote shell scripts that were implemented in embedded environments for automated recovery of lost devices
    \item Headed, completed, and support a project written in Python for automating part of a customer's testing workflow
    \item Built a wiki bot in Perl for automating most of my team's shift hand off process
\end{itemize}

\subsubsection*{Systems Administration}
\begin{itemize}
    \item Built and supported the company chat server using ejabberd with LDAP integration
    \item Identified a number of security holes in the legacy infrastructure and pushed them to fixes
    \item Administrator of the company wiki
\end{itemize}

\subsubsection*{Support Technician}
\begin{itemize}
    \item Worked out of ticketing systems including JIRA and Netsuite
    \item Administered RHEL Servers and embedded Linux devices
    \item Supported approximately 11000 servers and devices with a team of 10
\end{itemize}

\subsubsection*{Documentation}
\begin{itemize}
    \item Major contributor to the company wiki
    \item Assisted others in learning wiki markup and walked them through transferring documentation
    \item Created all the documentation for setting up company-provided laptops with a Linux distribution
\end{itemize}

\subsection*{Support Desk Year-Round Intern -- Hewlett Packard  \hfill June 2010-January 2012}
\subsubsection*{Development}
\begin{itemize}
    \item Wrote scripts to automate common tasks and gather information
    \item Redesigned existing legacy tools for compatibility with newer operating systems, and fixed a number of bugs
\end{itemize}

\subsubsection*{Support Technician}
\begin{itemize}
    \item Worked out of ticketing systems including GCSS for HP and BMC Remedy for the customer
    \item Supported servers in retail stores, distribution centers, and datacenters
    \item Administered HP-UX, Windows, Linux, and VMWare servers
    \item Worked with developers to track down problems
    \item Supported approximately 7000 servers with a team of 25
\end{itemize}

\subsubsection*{Documentation}
\begin{itemize}
    \item Created documentation on how to image clusters for a VMWare project
    \item Both created and moved documentation on to the team wiki
    \item Documented changes to my team's on-boarding process
\end{itemize}

\subsubsection*{Special Projects}
\begin{itemize}
    \item Helped refine the process for imaging VMWare clusters
    \item Assisted technicians with hardware installs
\end{itemize}

\subsubsection*{Networking}
\begin{itemize}
    \item Solved network issues including vlan memberships and switch configurations
    \item Worked regularly with Cisco network equipment
\end{itemize}

\section*{Education}
\subsection*{Rogers High School}
\begin{itemize}
    \item AP Java Programming class
    \item Various AP and dual-enrollment courses
\end{itemize}

\subsection*{Northwest Arkansas Community College}
\begin{itemize}
    \item Worked towards Associates degree in science
    \item Approximately 45 credit hours completed
\end{itemize}

\section*{References}
\begin{tabular}{l l l}
\textbf{Jon Duarte} & \textbf{Chad Wilson} & \textbf{Charles Witt} \\
Tier 2 Support, Consert & System Architect, Consert & NOC Lead, Consert\\
(210) 827-2777 & (832) 877-5333 & (210) 787-2734\\
\\
\textbf{Aaron Miller} & \textbf{Jared McGaugh} & \textbf{Joey Kavanaugh} \\
Developer, Couchbase & Technician, Hewlett-Packard & Developer, Hewlett-Packard\\
(479) 633-7625 & (479) 899-3515 & (479) 301-6814\\
\end{tabular}

\end{document}
